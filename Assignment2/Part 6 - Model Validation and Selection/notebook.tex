
% Default to the notebook output style

    


% Inherit from the specified cell style.




    
\documentclass[11pt]{article}

    
    
    \usepackage[T1]{fontenc}
    % Nicer default font (+ math font) than Computer Modern for most use cases
    \usepackage{mathpazo}

    % Basic figure setup, for now with no caption control since it's done
    % automatically by Pandoc (which extracts ![](path) syntax from Markdown).
    \usepackage{graphicx}
    % We will generate all images so they have a width \maxwidth. This means
    % that they will get their normal width if they fit onto the page, but
    % are scaled down if they would overflow the margins.
    \makeatletter
    \def\maxwidth{\ifdim\Gin@nat@width>\linewidth\linewidth
    \else\Gin@nat@width\fi}
    \makeatother
    \let\Oldincludegraphics\includegraphics
    % Set max figure width to be 80% of text width, for now hardcoded.
    \renewcommand{\includegraphics}[1]{\Oldincludegraphics[width=.8\maxwidth]{#1}}
    % Ensure that by default, figures have no caption (until we provide a
    % proper Figure object with a Caption API and a way to capture that
    % in the conversion process - todo).
    \usepackage{caption}
    \DeclareCaptionLabelFormat{nolabel}{}
    \captionsetup{labelformat=nolabel}

    \usepackage{adjustbox} % Used to constrain images to a maximum size 
    \usepackage{xcolor} % Allow colors to be defined
    \usepackage{enumerate} % Needed for markdown enumerations to work
    \usepackage{geometry} % Used to adjust the document margins
    \usepackage{amsmath} % Equations
    \usepackage{amssymb} % Equations
    \usepackage{textcomp} % defines textquotesingle
    % Hack from http://tex.stackexchange.com/a/47451/13684:
    \AtBeginDocument{%
        \def\PYZsq{\textquotesingle}% Upright quotes in Pygmentized code
    }
    \usepackage{upquote} % Upright quotes for verbatim code
    \usepackage{eurosym} % defines \euro
    \usepackage[mathletters]{ucs} % Extended unicode (utf-8) support
    \usepackage[utf8x]{inputenc} % Allow utf-8 characters in the tex document
    \usepackage{fancyvrb} % verbatim replacement that allows latex
    \usepackage{grffile} % extends the file name processing of package graphics 
                         % to support a larger range 
    % The hyperref package gives us a pdf with properly built
    % internal navigation ('pdf bookmarks' for the table of contents,
    % internal cross-reference links, web links for URLs, etc.)
    \usepackage{hyperref}
    \usepackage{longtable} % longtable support required by pandoc >1.10
    \usepackage{booktabs}  % table support for pandoc > 1.12.2
    \usepackage[inline]{enumitem} % IRkernel/repr support (it uses the enumerate* environment)
    \usepackage[normalem]{ulem} % ulem is needed to support strikethroughs (\sout)
                                % normalem makes italics be italics, not underlines
    

    
    
    % Colors for the hyperref package
    \definecolor{urlcolor}{rgb}{0,.145,.698}
    \definecolor{linkcolor}{rgb}{.71,0.21,0.01}
    \definecolor{citecolor}{rgb}{.12,.54,.11}

    % ANSI colors
    \definecolor{ansi-black}{HTML}{3E424D}
    \definecolor{ansi-black-intense}{HTML}{282C36}
    \definecolor{ansi-red}{HTML}{E75C58}
    \definecolor{ansi-red-intense}{HTML}{B22B31}
    \definecolor{ansi-green}{HTML}{00A250}
    \definecolor{ansi-green-intense}{HTML}{007427}
    \definecolor{ansi-yellow}{HTML}{DDB62B}
    \definecolor{ansi-yellow-intense}{HTML}{B27D12}
    \definecolor{ansi-blue}{HTML}{208FFB}
    \definecolor{ansi-blue-intense}{HTML}{0065CA}
    \definecolor{ansi-magenta}{HTML}{D160C4}
    \definecolor{ansi-magenta-intense}{HTML}{A03196}
    \definecolor{ansi-cyan}{HTML}{60C6C8}
    \definecolor{ansi-cyan-intense}{HTML}{258F8F}
    \definecolor{ansi-white}{HTML}{C5C1B4}
    \definecolor{ansi-white-intense}{HTML}{A1A6B2}

    % commands and environments needed by pandoc snippets
    % extracted from the output of `pandoc -s`
    \providecommand{\tightlist}{%
      \setlength{\itemsep}{0pt}\setlength{\parskip}{0pt}}
    \DefineVerbatimEnvironment{Highlighting}{Verbatim}{commandchars=\\\{\}}
    % Add ',fontsize=\small' for more characters per line
    \newenvironment{Shaded}{}{}
    \newcommand{\KeywordTok}[1]{\textcolor[rgb]{0.00,0.44,0.13}{\textbf{{#1}}}}
    \newcommand{\DataTypeTok}[1]{\textcolor[rgb]{0.56,0.13,0.00}{{#1}}}
    \newcommand{\DecValTok}[1]{\textcolor[rgb]{0.25,0.63,0.44}{{#1}}}
    \newcommand{\BaseNTok}[1]{\textcolor[rgb]{0.25,0.63,0.44}{{#1}}}
    \newcommand{\FloatTok}[1]{\textcolor[rgb]{0.25,0.63,0.44}{{#1}}}
    \newcommand{\CharTok}[1]{\textcolor[rgb]{0.25,0.44,0.63}{{#1}}}
    \newcommand{\StringTok}[1]{\textcolor[rgb]{0.25,0.44,0.63}{{#1}}}
    \newcommand{\CommentTok}[1]{\textcolor[rgb]{0.38,0.63,0.69}{\textit{{#1}}}}
    \newcommand{\OtherTok}[1]{\textcolor[rgb]{0.00,0.44,0.13}{{#1}}}
    \newcommand{\AlertTok}[1]{\textcolor[rgb]{1.00,0.00,0.00}{\textbf{{#1}}}}
    \newcommand{\FunctionTok}[1]{\textcolor[rgb]{0.02,0.16,0.49}{{#1}}}
    \newcommand{\RegionMarkerTok}[1]{{#1}}
    \newcommand{\ErrorTok}[1]{\textcolor[rgb]{1.00,0.00,0.00}{\textbf{{#1}}}}
    \newcommand{\NormalTok}[1]{{#1}}
    
    % Additional commands for more recent versions of Pandoc
    \newcommand{\ConstantTok}[1]{\textcolor[rgb]{0.53,0.00,0.00}{{#1}}}
    \newcommand{\SpecialCharTok}[1]{\textcolor[rgb]{0.25,0.44,0.63}{{#1}}}
    \newcommand{\VerbatimStringTok}[1]{\textcolor[rgb]{0.25,0.44,0.63}{{#1}}}
    \newcommand{\SpecialStringTok}[1]{\textcolor[rgb]{0.73,0.40,0.53}{{#1}}}
    \newcommand{\ImportTok}[1]{{#1}}
    \newcommand{\DocumentationTok}[1]{\textcolor[rgb]{0.73,0.13,0.13}{\textit{{#1}}}}
    \newcommand{\AnnotationTok}[1]{\textcolor[rgb]{0.38,0.63,0.69}{\textbf{\textit{{#1}}}}}
    \newcommand{\CommentVarTok}[1]{\textcolor[rgb]{0.38,0.63,0.69}{\textbf{\textit{{#1}}}}}
    \newcommand{\VariableTok}[1]{\textcolor[rgb]{0.10,0.09,0.49}{{#1}}}
    \newcommand{\ControlFlowTok}[1]{\textcolor[rgb]{0.00,0.44,0.13}{\textbf{{#1}}}}
    \newcommand{\OperatorTok}[1]{\textcolor[rgb]{0.40,0.40,0.40}{{#1}}}
    \newcommand{\BuiltInTok}[1]{{#1}}
    \newcommand{\ExtensionTok}[1]{{#1}}
    \newcommand{\PreprocessorTok}[1]{\textcolor[rgb]{0.74,0.48,0.00}{{#1}}}
    \newcommand{\AttributeTok}[1]{\textcolor[rgb]{0.49,0.56,0.16}{{#1}}}
    \newcommand{\InformationTok}[1]{\textcolor[rgb]{0.38,0.63,0.69}{\textbf{\textit{{#1}}}}}
    \newcommand{\WarningTok}[1]{\textcolor[rgb]{0.38,0.63,0.69}{\textbf{\textit{{#1}}}}}
    
    
    % Define a nice break command that doesn't care if a line doesn't already
    % exist.
    \def\br{\hspace*{\fill} \\* }
    % Math Jax compatability definitions
    \def\gt{>}
    \def\lt{<}
    % Document parameters
    \title{Model Validation and Selection}
    
    
    

    % Pygments definitions
    
\makeatletter
\def\PY@reset{\let\PY@it=\relax \let\PY@bf=\relax%
    \let\PY@ul=\relax \let\PY@tc=\relax%
    \let\PY@bc=\relax \let\PY@ff=\relax}
\def\PY@tok#1{\csname PY@tok@#1\endcsname}
\def\PY@toks#1+{\ifx\relax#1\empty\else%
    \PY@tok{#1}\expandafter\PY@toks\fi}
\def\PY@do#1{\PY@bc{\PY@tc{\PY@ul{%
    \PY@it{\PY@bf{\PY@ff{#1}}}}}}}
\def\PY#1#2{\PY@reset\PY@toks#1+\relax+\PY@do{#2}}

\expandafter\def\csname PY@tok@w\endcsname{\def\PY@tc##1{\textcolor[rgb]{0.73,0.73,0.73}{##1}}}
\expandafter\def\csname PY@tok@c\endcsname{\let\PY@it=\textit\def\PY@tc##1{\textcolor[rgb]{0.25,0.50,0.50}{##1}}}
\expandafter\def\csname PY@tok@cp\endcsname{\def\PY@tc##1{\textcolor[rgb]{0.74,0.48,0.00}{##1}}}
\expandafter\def\csname PY@tok@k\endcsname{\let\PY@bf=\textbf\def\PY@tc##1{\textcolor[rgb]{0.00,0.50,0.00}{##1}}}
\expandafter\def\csname PY@tok@kp\endcsname{\def\PY@tc##1{\textcolor[rgb]{0.00,0.50,0.00}{##1}}}
\expandafter\def\csname PY@tok@kt\endcsname{\def\PY@tc##1{\textcolor[rgb]{0.69,0.00,0.25}{##1}}}
\expandafter\def\csname PY@tok@o\endcsname{\def\PY@tc##1{\textcolor[rgb]{0.40,0.40,0.40}{##1}}}
\expandafter\def\csname PY@tok@ow\endcsname{\let\PY@bf=\textbf\def\PY@tc##1{\textcolor[rgb]{0.67,0.13,1.00}{##1}}}
\expandafter\def\csname PY@tok@nb\endcsname{\def\PY@tc##1{\textcolor[rgb]{0.00,0.50,0.00}{##1}}}
\expandafter\def\csname PY@tok@nf\endcsname{\def\PY@tc##1{\textcolor[rgb]{0.00,0.00,1.00}{##1}}}
\expandafter\def\csname PY@tok@nc\endcsname{\let\PY@bf=\textbf\def\PY@tc##1{\textcolor[rgb]{0.00,0.00,1.00}{##1}}}
\expandafter\def\csname PY@tok@nn\endcsname{\let\PY@bf=\textbf\def\PY@tc##1{\textcolor[rgb]{0.00,0.00,1.00}{##1}}}
\expandafter\def\csname PY@tok@ne\endcsname{\let\PY@bf=\textbf\def\PY@tc##1{\textcolor[rgb]{0.82,0.25,0.23}{##1}}}
\expandafter\def\csname PY@tok@nv\endcsname{\def\PY@tc##1{\textcolor[rgb]{0.10,0.09,0.49}{##1}}}
\expandafter\def\csname PY@tok@no\endcsname{\def\PY@tc##1{\textcolor[rgb]{0.53,0.00,0.00}{##1}}}
\expandafter\def\csname PY@tok@nl\endcsname{\def\PY@tc##1{\textcolor[rgb]{0.63,0.63,0.00}{##1}}}
\expandafter\def\csname PY@tok@ni\endcsname{\let\PY@bf=\textbf\def\PY@tc##1{\textcolor[rgb]{0.60,0.60,0.60}{##1}}}
\expandafter\def\csname PY@tok@na\endcsname{\def\PY@tc##1{\textcolor[rgb]{0.49,0.56,0.16}{##1}}}
\expandafter\def\csname PY@tok@nt\endcsname{\let\PY@bf=\textbf\def\PY@tc##1{\textcolor[rgb]{0.00,0.50,0.00}{##1}}}
\expandafter\def\csname PY@tok@nd\endcsname{\def\PY@tc##1{\textcolor[rgb]{0.67,0.13,1.00}{##1}}}
\expandafter\def\csname PY@tok@s\endcsname{\def\PY@tc##1{\textcolor[rgb]{0.73,0.13,0.13}{##1}}}
\expandafter\def\csname PY@tok@sd\endcsname{\let\PY@it=\textit\def\PY@tc##1{\textcolor[rgb]{0.73,0.13,0.13}{##1}}}
\expandafter\def\csname PY@tok@si\endcsname{\let\PY@bf=\textbf\def\PY@tc##1{\textcolor[rgb]{0.73,0.40,0.53}{##1}}}
\expandafter\def\csname PY@tok@se\endcsname{\let\PY@bf=\textbf\def\PY@tc##1{\textcolor[rgb]{0.73,0.40,0.13}{##1}}}
\expandafter\def\csname PY@tok@sr\endcsname{\def\PY@tc##1{\textcolor[rgb]{0.73,0.40,0.53}{##1}}}
\expandafter\def\csname PY@tok@ss\endcsname{\def\PY@tc##1{\textcolor[rgb]{0.10,0.09,0.49}{##1}}}
\expandafter\def\csname PY@tok@sx\endcsname{\def\PY@tc##1{\textcolor[rgb]{0.00,0.50,0.00}{##1}}}
\expandafter\def\csname PY@tok@m\endcsname{\def\PY@tc##1{\textcolor[rgb]{0.40,0.40,0.40}{##1}}}
\expandafter\def\csname PY@tok@gh\endcsname{\let\PY@bf=\textbf\def\PY@tc##1{\textcolor[rgb]{0.00,0.00,0.50}{##1}}}
\expandafter\def\csname PY@tok@gu\endcsname{\let\PY@bf=\textbf\def\PY@tc##1{\textcolor[rgb]{0.50,0.00,0.50}{##1}}}
\expandafter\def\csname PY@tok@gd\endcsname{\def\PY@tc##1{\textcolor[rgb]{0.63,0.00,0.00}{##1}}}
\expandafter\def\csname PY@tok@gi\endcsname{\def\PY@tc##1{\textcolor[rgb]{0.00,0.63,0.00}{##1}}}
\expandafter\def\csname PY@tok@gr\endcsname{\def\PY@tc##1{\textcolor[rgb]{1.00,0.00,0.00}{##1}}}
\expandafter\def\csname PY@tok@ge\endcsname{\let\PY@it=\textit}
\expandafter\def\csname PY@tok@gs\endcsname{\let\PY@bf=\textbf}
\expandafter\def\csname PY@tok@gp\endcsname{\let\PY@bf=\textbf\def\PY@tc##1{\textcolor[rgb]{0.00,0.00,0.50}{##1}}}
\expandafter\def\csname PY@tok@go\endcsname{\def\PY@tc##1{\textcolor[rgb]{0.53,0.53,0.53}{##1}}}
\expandafter\def\csname PY@tok@gt\endcsname{\def\PY@tc##1{\textcolor[rgb]{0.00,0.27,0.87}{##1}}}
\expandafter\def\csname PY@tok@err\endcsname{\def\PY@bc##1{\setlength{\fboxsep}{0pt}\fcolorbox[rgb]{1.00,0.00,0.00}{1,1,1}{\strut ##1}}}
\expandafter\def\csname PY@tok@kc\endcsname{\let\PY@bf=\textbf\def\PY@tc##1{\textcolor[rgb]{0.00,0.50,0.00}{##1}}}
\expandafter\def\csname PY@tok@kd\endcsname{\let\PY@bf=\textbf\def\PY@tc##1{\textcolor[rgb]{0.00,0.50,0.00}{##1}}}
\expandafter\def\csname PY@tok@kn\endcsname{\let\PY@bf=\textbf\def\PY@tc##1{\textcolor[rgb]{0.00,0.50,0.00}{##1}}}
\expandafter\def\csname PY@tok@kr\endcsname{\let\PY@bf=\textbf\def\PY@tc##1{\textcolor[rgb]{0.00,0.50,0.00}{##1}}}
\expandafter\def\csname PY@tok@bp\endcsname{\def\PY@tc##1{\textcolor[rgb]{0.00,0.50,0.00}{##1}}}
\expandafter\def\csname PY@tok@fm\endcsname{\def\PY@tc##1{\textcolor[rgb]{0.00,0.00,1.00}{##1}}}
\expandafter\def\csname PY@tok@vc\endcsname{\def\PY@tc##1{\textcolor[rgb]{0.10,0.09,0.49}{##1}}}
\expandafter\def\csname PY@tok@vg\endcsname{\def\PY@tc##1{\textcolor[rgb]{0.10,0.09,0.49}{##1}}}
\expandafter\def\csname PY@tok@vi\endcsname{\def\PY@tc##1{\textcolor[rgb]{0.10,0.09,0.49}{##1}}}
\expandafter\def\csname PY@tok@vm\endcsname{\def\PY@tc##1{\textcolor[rgb]{0.10,0.09,0.49}{##1}}}
\expandafter\def\csname PY@tok@sa\endcsname{\def\PY@tc##1{\textcolor[rgb]{0.73,0.13,0.13}{##1}}}
\expandafter\def\csname PY@tok@sb\endcsname{\def\PY@tc##1{\textcolor[rgb]{0.73,0.13,0.13}{##1}}}
\expandafter\def\csname PY@tok@sc\endcsname{\def\PY@tc##1{\textcolor[rgb]{0.73,0.13,0.13}{##1}}}
\expandafter\def\csname PY@tok@dl\endcsname{\def\PY@tc##1{\textcolor[rgb]{0.73,0.13,0.13}{##1}}}
\expandafter\def\csname PY@tok@s2\endcsname{\def\PY@tc##1{\textcolor[rgb]{0.73,0.13,0.13}{##1}}}
\expandafter\def\csname PY@tok@sh\endcsname{\def\PY@tc##1{\textcolor[rgb]{0.73,0.13,0.13}{##1}}}
\expandafter\def\csname PY@tok@s1\endcsname{\def\PY@tc##1{\textcolor[rgb]{0.73,0.13,0.13}{##1}}}
\expandafter\def\csname PY@tok@mb\endcsname{\def\PY@tc##1{\textcolor[rgb]{0.40,0.40,0.40}{##1}}}
\expandafter\def\csname PY@tok@mf\endcsname{\def\PY@tc##1{\textcolor[rgb]{0.40,0.40,0.40}{##1}}}
\expandafter\def\csname PY@tok@mh\endcsname{\def\PY@tc##1{\textcolor[rgb]{0.40,0.40,0.40}{##1}}}
\expandafter\def\csname PY@tok@mi\endcsname{\def\PY@tc##1{\textcolor[rgb]{0.40,0.40,0.40}{##1}}}
\expandafter\def\csname PY@tok@il\endcsname{\def\PY@tc##1{\textcolor[rgb]{0.40,0.40,0.40}{##1}}}
\expandafter\def\csname PY@tok@mo\endcsname{\def\PY@tc##1{\textcolor[rgb]{0.40,0.40,0.40}{##1}}}
\expandafter\def\csname PY@tok@ch\endcsname{\let\PY@it=\textit\def\PY@tc##1{\textcolor[rgb]{0.25,0.50,0.50}{##1}}}
\expandafter\def\csname PY@tok@cm\endcsname{\let\PY@it=\textit\def\PY@tc##1{\textcolor[rgb]{0.25,0.50,0.50}{##1}}}
\expandafter\def\csname PY@tok@cpf\endcsname{\let\PY@it=\textit\def\PY@tc##1{\textcolor[rgb]{0.25,0.50,0.50}{##1}}}
\expandafter\def\csname PY@tok@c1\endcsname{\let\PY@it=\textit\def\PY@tc##1{\textcolor[rgb]{0.25,0.50,0.50}{##1}}}
\expandafter\def\csname PY@tok@cs\endcsname{\let\PY@it=\textit\def\PY@tc##1{\textcolor[rgb]{0.25,0.50,0.50}{##1}}}

\def\PYZbs{\char`\\}
\def\PYZus{\char`\_}
\def\PYZob{\char`\{}
\def\PYZcb{\char`\}}
\def\PYZca{\char`\^}
\def\PYZam{\char`\&}
\def\PYZlt{\char`\<}
\def\PYZgt{\char`\>}
\def\PYZsh{\char`\#}
\def\PYZpc{\char`\%}
\def\PYZdl{\char`\$}
\def\PYZhy{\char`\-}
\def\PYZsq{\char`\'}
\def\PYZdq{\char`\"}
\def\PYZti{\char`\~}
% for compatibility with earlier versions
\def\PYZat{@}
\def\PYZlb{[}
\def\PYZrb{]}
\makeatother


    % Exact colors from NB
    \definecolor{incolor}{rgb}{0.0, 0.0, 0.5}
    \definecolor{outcolor}{rgb}{0.545, 0.0, 0.0}



    
    % Prevent overflowing lines due to hard-to-break entities
    \sloppy 
    % Setup hyperref package
    \hypersetup{
      breaklinks=true,  % so long urls are correctly broken across lines
      colorlinks=true,
      urlcolor=urlcolor,
      linkcolor=linkcolor,
      citecolor=citecolor,
      }
    % Slightly bigger margins than the latex defaults
    
    \geometry{verbose,tmargin=1in,bmargin=1in,lmargin=1in,rmargin=1in}
    
    

    \begin{document}
    
    
    \maketitle
    
    

    
    \section{Model Validation and
Selection}\label{model-validation-and-selection}

    In this notebook we will be going through varies methods to select a
model based upon the following features

    \begin{verbatim}
<li>cross validation techniques</li>
<li>Bias -variance tradeoff</li>
<li>regularization (L1, L2, Elastic net)</li>
<li>grid search options.</li>
\end{verbatim}

    \subsection{Definition of Validation}\label{definition-of-validation}

    \subsubsection{cross validation
techniques}\label{cross-validation-techniques}

    Cross-validation is a technique for evaluating ML models by training
several ML models on subsets of the available input data and evaluating
them on the complementary subset of the data. Use cross-validation to
detect overfitting, ie, failing to generalize a pattern.

In ML, you can use the k-fold cross-validation method to perform
cross-validation. In k-fold cross-validation, you split the input data
into k subsets of data (also known as folds). You train an ML model on
all but one (k-1) of the subsets, and then evaluate the model on the
subset that was not used for training. This process is repeated k times,
with a different subset reserved for evaluation (and excluded from
training) each time.

    \begin{Verbatim}[commandchars=\\\{\}]
{\color{incolor}In [{\color{incolor}1}]:} \PY{k+kn}{import} \PY{n+nn}{pandas} \PY{k}{as} \PY{n+nn}{pd}
        \PY{k+kn}{import} \PY{n+nn}{numpy} \PY{k}{as} \PY{n+nn}{np}
        \PY{k+kn}{import} \PY{n+nn}{seaborn} \PY{k}{as} \PY{n+nn}{sns}
        \PY{k+kn}{import} \PY{n+nn}{matplotlib}\PY{n+nn}{.}\PY{n+nn}{pyplot} \PY{k}{as} \PY{n+nn}{plt}
        \PY{k+kn}{from} \PY{n+nn}{sklearn}\PY{n+nn}{.}\PY{n+nn}{model\PYZus{}selection} \PY{k}{import} \PY{n}{train\PYZus{}test\PYZus{}split}
        \PY{o}{\PYZpc{}}\PY{k}{matplotlib} inline
\end{Verbatim}


    \begin{Verbatim}[commandchars=\\\{\}]
{\color{incolor}In [{\color{incolor}8}]:} \PY{n}{data}\PY{o}{=} \PY{n}{pd}\PY{o}{.}\PY{n}{read\PYZus{}csv}\PY{p}{(}\PY{l+s+s2}{\PYZdq{}}\PY{l+s+s2}{./Data/new\PYZus{}data\PYZus{}feature.csv}\PY{l+s+s2}{\PYZdq{}}\PY{p}{)}
        \PY{n}{data} \PY{o}{=} \PY{n}{pd}\PY{o}{.}\PY{n}{get\PYZus{}dummies}\PY{p}{(}\PY{n}{data}\PY{p}{,} \PY{n}{columns}\PY{o}{=}\PY{p}{[}\PY{l+s+s2}{\PYZdq{}}\PY{l+s+s2}{timeofDay}\PY{l+s+s2}{\PYZdq{}}\PY{p}{,}\PY{l+s+s2}{\PYZdq{}}\PY{l+s+s2}{activeStatus}\PY{l+s+s2}{\PYZdq{}}\PY{p}{,}\PY{l+s+s1}{\PYZsq{}}\PY{l+s+s1}{dayoftheweek}\PY{l+s+s1}{\PYZsq{}}\PY{p}{,}\PY{l+s+s1}{\PYZsq{}}\PY{l+s+s1}{WeekDayType}\PY{l+s+s1}{\PYZsq{}}\PY{p}{,}\PY{l+s+s1}{\PYZsq{}}\PY{l+s+s1}{season}\PY{l+s+s1}{\PYZsq{}}\PY{p}{]}\PY{p}{)}
\end{Verbatim}


    \begin{Verbatim}[commandchars=\\\{\}]
{\color{incolor}In [{\color{incolor}9}]:} \PY{c+c1}{\PYZsh{}Select response y and Training set X}
        \PY{n}{k} \PY{o}{=} \PY{n+nb}{list}\PY{p}{(}\PY{n}{data}\PY{p}{)}
        \PY{n}{k}
        \PY{n}{my\PYZus{}cols} \PY{o}{=} \PY{p}{[}
         \PY{l+s+s1}{\PYZsq{}}\PY{l+s+s1}{lights}\PY{l+s+s1}{\PYZsq{}}\PY{p}{,}
         \PY{l+s+s1}{\PYZsq{}}\PY{l+s+s1}{T1}\PY{l+s+s1}{\PYZsq{}}\PY{p}{,}
         \PY{l+s+s1}{\PYZsq{}}\PY{l+s+s1}{RH\PYZus{}1}\PY{l+s+s1}{\PYZsq{}}\PY{p}{,}
         \PY{l+s+s1}{\PYZsq{}}\PY{l+s+s1}{T2}\PY{l+s+s1}{\PYZsq{}}\PY{p}{,}
         \PY{l+s+s1}{\PYZsq{}}\PY{l+s+s1}{RH\PYZus{}2}\PY{l+s+s1}{\PYZsq{}}\PY{p}{,}
         \PY{l+s+s1}{\PYZsq{}}\PY{l+s+s1}{T3}\PY{l+s+s1}{\PYZsq{}}\PY{p}{,}
         \PY{l+s+s1}{\PYZsq{}}\PY{l+s+s1}{RH\PYZus{}3}\PY{l+s+s1}{\PYZsq{}}\PY{p}{,}
         \PY{l+s+s1}{\PYZsq{}}\PY{l+s+s1}{T4}\PY{l+s+s1}{\PYZsq{}}\PY{p}{,}
         \PY{l+s+s1}{\PYZsq{}}\PY{l+s+s1}{RH\PYZus{}4}\PY{l+s+s1}{\PYZsq{}}\PY{p}{,}
         \PY{l+s+s1}{\PYZsq{}}\PY{l+s+s1}{T5}\PY{l+s+s1}{\PYZsq{}}\PY{p}{,}
         \PY{l+s+s1}{\PYZsq{}}\PY{l+s+s1}{RH\PYZus{}5}\PY{l+s+s1}{\PYZsq{}}\PY{p}{,}
         \PY{l+s+s1}{\PYZsq{}}\PY{l+s+s1}{T6}\PY{l+s+s1}{\PYZsq{}}\PY{p}{,}
         \PY{l+s+s1}{\PYZsq{}}\PY{l+s+s1}{RH\PYZus{}6}\PY{l+s+s1}{\PYZsq{}}\PY{p}{,}
         \PY{l+s+s1}{\PYZsq{}}\PY{l+s+s1}{T7}\PY{l+s+s1}{\PYZsq{}}\PY{p}{,}
         \PY{l+s+s1}{\PYZsq{}}\PY{l+s+s1}{RH\PYZus{}7}\PY{l+s+s1}{\PYZsq{}}\PY{p}{,}
         \PY{l+s+s1}{\PYZsq{}}\PY{l+s+s1}{T8}\PY{l+s+s1}{\PYZsq{}}\PY{p}{,}
         \PY{l+s+s1}{\PYZsq{}}\PY{l+s+s1}{RH\PYZus{}8}\PY{l+s+s1}{\PYZsq{}}\PY{p}{,}
         \PY{l+s+s1}{\PYZsq{}}\PY{l+s+s1}{T9}\PY{l+s+s1}{\PYZsq{}}\PY{p}{,}
         \PY{l+s+s1}{\PYZsq{}}\PY{l+s+s1}{RH\PYZus{}9}\PY{l+s+s1}{\PYZsq{}}\PY{p}{,}
         \PY{l+s+s1}{\PYZsq{}}\PY{l+s+s1}{T\PYZus{}out}\PY{l+s+s1}{\PYZsq{}}\PY{p}{,}
         \PY{l+s+s1}{\PYZsq{}}\PY{l+s+s1}{Press\PYZus{}mm\PYZus{}hg}\PY{l+s+s1}{\PYZsq{}}\PY{p}{,}
         \PY{l+s+s1}{\PYZsq{}}\PY{l+s+s1}{RH\PYZus{}out}\PY{l+s+s1}{\PYZsq{}}\PY{p}{,}
         \PY{l+s+s1}{\PYZsq{}}\PY{l+s+s1}{Windspeed}\PY{l+s+s1}{\PYZsq{}}\PY{p}{,}
         \PY{l+s+s1}{\PYZsq{}}\PY{l+s+s1}{Visibility}\PY{l+s+s1}{\PYZsq{}}\PY{p}{,}
         \PY{l+s+s1}{\PYZsq{}}\PY{l+s+s1}{Tdewpoint}\PY{l+s+s1}{\PYZsq{}}\PY{p}{,}
         \PY{l+s+s1}{\PYZsq{}}\PY{l+s+s1}{NSM}\PY{l+s+s1}{\PYZsq{}}\PY{p}{,}
         \PY{l+s+s1}{\PYZsq{}}\PY{l+s+s1}{weekOfTheYear}\PY{l+s+s1}{\PYZsq{}}\PY{p}{,}
         \PY{l+s+s1}{\PYZsq{}}\PY{l+s+s1}{timeofDay\PYZus{}Day}\PY{l+s+s1}{\PYZsq{}}\PY{p}{,}
         \PY{l+s+s1}{\PYZsq{}}\PY{l+s+s1}{timeofDay\PYZus{}Night}\PY{l+s+s1}{\PYZsq{}}\PY{p}{,}
         \PY{l+s+s1}{\PYZsq{}}\PY{l+s+s1}{activeStatus\PYZus{}awake}\PY{l+s+s1}{\PYZsq{}}\PY{p}{,}
         \PY{l+s+s1}{\PYZsq{}}\PY{l+s+s1}{activeStatus\PYZus{}sleep}\PY{l+s+s1}{\PYZsq{}}\PY{p}{,}
         \PY{l+s+s1}{\PYZsq{}}\PY{l+s+s1}{dayoftheweek\PYZus{}Friday}\PY{l+s+s1}{\PYZsq{}}\PY{p}{,}
         \PY{l+s+s1}{\PYZsq{}}\PY{l+s+s1}{dayoftheweek\PYZus{}Monday}\PY{l+s+s1}{\PYZsq{}}\PY{p}{,}
         \PY{l+s+s1}{\PYZsq{}}\PY{l+s+s1}{dayoftheweek\PYZus{}Saturday}\PY{l+s+s1}{\PYZsq{}}\PY{p}{,}
         \PY{l+s+s1}{\PYZsq{}}\PY{l+s+s1}{dayoftheweek\PYZus{}Sunday}\PY{l+s+s1}{\PYZsq{}}\PY{p}{,}
         \PY{l+s+s1}{\PYZsq{}}\PY{l+s+s1}{dayoftheweek\PYZus{}Thurday}\PY{l+s+s1}{\PYZsq{}}\PY{p}{,}
         \PY{l+s+s1}{\PYZsq{}}\PY{l+s+s1}{dayoftheweek\PYZus{}Tuesday}\PY{l+s+s1}{\PYZsq{}}\PY{p}{,}
         \PY{l+s+s1}{\PYZsq{}}\PY{l+s+s1}{dayoftheweek\PYZus{}Wednesday}\PY{l+s+s1}{\PYZsq{}}\PY{p}{,}
         \PY{l+s+s1}{\PYZsq{}}\PY{l+s+s1}{WeekDayType\PYZus{}Weekday}\PY{l+s+s1}{\PYZsq{}}\PY{p}{,}
         \PY{l+s+s1}{\PYZsq{}}\PY{l+s+s1}{WeekDayType\PYZus{}weekend}\PY{l+s+s1}{\PYZsq{}}\PY{p}{,}
         \PY{l+s+s1}{\PYZsq{}}\PY{l+s+s1}{season\PYZus{}spring}\PY{l+s+s1}{\PYZsq{}}\PY{p}{,}
         \PY{l+s+s1}{\PYZsq{}}\PY{l+s+s1}{season\PYZus{}winter}\PY{l+s+s1}{\PYZsq{}}\PY{p}{]}
        \PY{n}{X} \PY{o}{=} \PY{n}{data}\PY{p}{[}\PY{n}{my\PYZus{}cols}\PY{p}{]}\PY{o}{.}\PY{n}{values}
        \PY{n}{y} \PY{o}{=} \PY{n}{data}\PY{p}{[}\PY{l+s+s1}{\PYZsq{}}\PY{l+s+s1}{Appliances}\PY{l+s+s1}{\PYZsq{}}\PY{p}{]}
\end{Verbatim}


    \begin{Verbatim}[commandchars=\\\{\}]
{\color{incolor}In [{\color{incolor}10}]:} \PY{n}{X} \PY{o}{=} \PY{n}{data}\PY{p}{[}\PY{n}{my\PYZus{}cols}\PY{p}{]}
         \PY{n}{y} \PY{o}{=} \PY{n}{data}\PY{p}{[}\PY{l+s+s1}{\PYZsq{}}\PY{l+s+s1}{Appliances}\PY{l+s+s1}{\PYZsq{}}\PY{p}{]}
         \PY{c+c1}{\PYZsh{} min\PYZus{}max\PYZus{}scaler = MinMaxScaler()}
         \PY{c+c1}{\PYZsh{} X = min\PYZus{}max\PYZus{}scaler.fit\PYZus{}transform(X)}
         \PY{c+c1}{\PYZsh{} y = min\PYZus{}max\PYZus{}scaler.fit\PYZus{}transform(y.reshape(1, \PYZhy{}1))}
         \PY{n}{X\PYZus{}train}\PY{p}{,} \PY{n}{X\PYZus{}test}\PY{p}{,} \PY{n}{y\PYZus{}train}\PY{p}{,} \PY{n}{y\PYZus{}test} \PY{o}{=} \PY{n}{train\PYZus{}test\PYZus{}split}\PY{p}{(}\PY{n}{X}\PY{p}{,} \PY{n}{y}\PY{p}{,}\PY{n}{train\PYZus{}size}\PY{o}{=}\PY{l+m+mf}{0.75}\PY{p}{,} \PY{n}{test\PYZus{}size}\PY{o}{=}\PY{l+m+mf}{0.25}\PY{p}{)}
\end{Verbatim}


    \begin{Verbatim}[commandchars=\\\{\}]
{\color{incolor}In [{\color{incolor}12}]:} \PY{k+kn}{from} \PY{n+nn}{sklearn}\PY{n+nn}{.}\PY{n+nn}{ensemble} \PY{k}{import} \PY{n}{RandomForestRegressor}
         \PY{k+kn}{from} \PY{n+nn}{math} \PY{k}{import} \PY{n}{sqrt}
         \PY{k+kn}{from} \PY{n+nn}{sklearn}\PY{n+nn}{.}\PY{n+nn}{metrics} \PY{k}{import} \PY{n}{mean\PYZus{}squared\PYZus{}error}
         \PY{k+kn}{from} \PY{n+nn}{sklearn}\PY{n+nn}{.}\PY{n+nn}{metrics} \PY{k}{import} \PY{n}{r2\PYZus{}score}
         \PY{k+kn}{from} \PY{n+nn}{sklearn}\PY{n+nn}{.}\PY{n+nn}{metrics} \PY{k}{import} \PY{n}{mean\PYZus{}absolute\PYZus{}error}
\end{Verbatim}


    \begin{Verbatim}[commandchars=\\\{\}]
{\color{incolor}In [{\color{incolor}14}]:} \PY{n}{rf} \PY{o}{=} \PY{n}{RandomForestRegressor}\PY{p}{(}\PY{n}{n\PYZus{}estimators} \PY{o}{=} \PY{l+m+mi}{300}\PY{p}{)}
\end{Verbatim}


    \begin{Verbatim}[commandchars=\\\{\}]
{\color{incolor}In [{\color{incolor} }]:} \PY{k+kn}{from} \PY{n+nn}{sklearn}\PY{n+nn}{.}\PY{n+nn}{cross\PYZus{}validation} \PY{k}{import} \PY{n}{cross\PYZus{}val\PYZus{}score}
        \PY{n}{scores} \PY{o}{=} \PY{n}{cross\PYZus{}val\PYZus{}score}\PY{p}{(}\PY{n}{rf}\PY{p}{,} \PY{n}{X}\PY{p}{,} \PY{n}{y}\PY{p}{,} \PY{n}{cv}\PY{o}{=}\PY{l+m+mi}{5}\PY{p}{)}
        \PY{n+nb}{print}\PY{p}{(}\PY{l+s+s2}{\PYZdq{}}\PY{l+s+s2}{Accuracy: }\PY{l+s+si}{\PYZpc{}0.2f}\PY{l+s+s2}{ (+/\PYZhy{} }\PY{l+s+si}{\PYZpc{}0.2f}\PY{l+s+s2}{)}\PY{l+s+s2}{\PYZdq{}} \PY{o}{\PYZpc{}} \PY{p}{(}\PY{n}{scores}\PY{o}{.}\PY{n}{mean}\PY{p}{(}\PY{p}{)}\PY{p}{,} \PY{n}{scores}\PY{o}{.}\PY{n}{std}\PY{p}{(}\PY{p}{)} \PY{o}{*} \PY{l+m+mi}{2}\PY{p}{)}\PY{p}{)}
\end{Verbatim}


    \subsubsection{Bias -variance tradeoff}\label{bias--variance-tradeoff}

    Overview of Bias and Variance In supervised machine learning an
algorithm learns a model from training data.

The goal of any supervised machine learning algorithm is to best
estimate the mapping function (f) for the output variable (Y) given the
input data (X). The mapping function is often called the target function
because it is the function that a given supervised machine learning
algorithm aims to approximate.

The prediction error for any machine learning algorithm can be broken
down into three parts:

~

\begin{verbatim}
<li>Bias Error</li>
<li>Variance Error</li>
<li>Irreducible Error</li>
\end{verbatim}

~

The irreducible error cannot be reduced regardless of what algorithm is
used. It is the error introduced from the chosen framing of the problem
and may be caused by factors like unknown variables that influence the
mapping of the input variables to the output variable.

    Bias Error

Bias are the simplifying assumptions made by a model to make the target
function easier to learn.

Generally, parametric algorithms have a high bias making them fast to
learn and easier to understand but generally less flexible. In turn,
they~have lower predictive performance on complex problems that fail to
meet the simplifying assumptions of the algorithms bias.

\begin{verbatim}
<li><strong>Low Bias</strong>: Suggests less assumptions about the form of the target function.</li>
<li><strong>High-Bias</strong>:&nbsp;Suggests more assumptions about the form of the target function.</li>
\end{verbatim}

Examples of low-bias machine learning algorithms~include:~Decision
Trees,~k-Nearest Neighbors and~Support Vector Machines.

Examples of high-bias machine learning algorithms include:~Linear
Regression,~Linear Discriminant Analysis and~Logistic Regression.

Variance Error

Variance is the amount that the estimate of the target function will
change if different training data was used.

The target function is estimated from the training data by a machine
learning algorithm, so we should expect the algorithm to have some
variance. Ideally, it should not change too much from one training
dataset to the next, meaning that the algorithm is good at picking out
the hidden underlying mapping between the inputs and the output
variables.

Machine learning algorithms that have a high variance are strongly
influenced by the specifics of the training data. This means that the
specifics of the training have~influences the number and types of
parameters used to characterize the mapping function.

\begin{verbatim}
<li><strong>Low Variance</strong>: Suggests small changes to the estimate of the target function with changes to the training dataset.</li>
<li><strong>High Variance</strong>: Suggests large changes to the estimate of the target function with changes to the training dataset.</li>
\end{verbatim}

Generally, nonparametric machine learning algorithms that have a lot of
flexibility have a high variance. For example, decision trees have a
high variance, that is even higher if the trees are not pruned before
use.

Examples of low-variance machine learning algorithms include:~Linear
Regression,~Linear Discriminant Analysis and~Logistic Regression.

Examples of high-variance machine learning algorithms include:~Decision
Trees,~k-Nearest Neighbors and~Support Vector Machines.

Bias-Variance Trade-Off

The goal of any supervised machine learning algorithm is to achieve low
bias and low variance. In turn the algorithm should achieve good
prediction performance.

You can see a general trend in the examples above:

\begin{verbatim}
<li>Parametric or linear machine learning algorithms often have a high bias but a low variance.</li>
<li>Non-parametric or non-linear machine learning algorithms often have a low bias but a high variance.</li>
\end{verbatim}

The parameterization of machine learning algorithms is often a battle to
balance out bias and variance.

Below are two examples of configuring the bias-variance trade-off for
specific algorithms:

\begin{verbatim}
<li>The k-nearest neighbors algorithm has low bias and high variance, but the trade-off can be changed by increasing the value of k which increases the number of neighbors that contribute t the prediction and in turn increases the bias of the model.</li>
<li>The support vector machine algorithm has low bias and high variance, but the trade-off can be changed by increasing the C parameter that influences the number of violations of the margin allowed in the training data which increases the bias but decreases the variance.</li>
\end{verbatim}

There is no escaping~the relationship between bias and variance in
machine learning.

\begin{verbatim}
<li>Increasing the bias will decrease the variance.</li>
<li>Increasing the variance will decrease the bias.</li>
\end{verbatim}

There is a trade-off at play between these two concerns and the
algorithms you choose and the way you choose to configure them are
finding different balances in this trade-off for your problem

In reality, we cannot calculate the real bias and variance error terms
because we do not know the actual underlying target function.
Nevertheless, as a framework, bias and variance provide the tools to
understand the behavior of machine learning algorithms in the pursuit of
predictive performance.

    \subsubsection{Regularization (L1, L2, Elastic
net)}\label{regularization-l1-l2-elastic-net}

    Regularization is used to fix over fitting of models

    Regularization adds a penalty on the different parameters of the model
to reduce the freedom of the model. Hence, the model will be~less likely
to fit the noise~of the training data and will improve the
generalization abilities of the model. We will study and compare:

\begin{verbatim}
<li>The L1 regularization (also called Lasso)</li>
<li>The L2 regularization (also called Ridge)</li>
<li>The L1/L2 regularization (also called Elastic net)</li>
\end{verbatim}

You can find the R code for regularization at the end of the post.

L1 Regularization (Lasso penalisation)

The L1 regularization adds a penalty equal to the sum of the absolute
value of the coefficients.~The L1 regularization will~shrink some
parameters to zero. Hence some variables will not play any role in the
model, L1 regression can be seen as a way to select~features in a
model.~

~As lambda grows bigger, more coefficient will be cut. Below is the
evolution of the value of the different coefficients while lambda is
growing.

As expected, coefficients are cut one by one until no variables remain.
Let's see how the test error is evolving:

At the beginning, cutting coefficient reduces the overfitting and the
generalization abilities of the model. Hence, the test error is
decreasing. However, as we are cutting more and more coefficient, the
test error start increasing. The model is not able to learn complex
pattern with so few variables.

~

~

    L2 Regularization (Ridge penalisation)

The L2 regularization adds a penalty equal to the sum of the squared
value of the coefficients.

The L2 regularization will force~the parameters to be relatively small,
the bigger the penalization, the smaller (and the more robust) the
coefficients are.

When we compare this plot to the L1 regularization plot, we notice that
the coefficients decrease progressively and are not cut to zero. They
slowly decrease to zero. That is the behavior we expected. Let's see how
the test error evolves:

~

    Elastic-net

Elastic-net is a mix of~both L1 and L2 regularizations. A penalty is
applied to the sum of the absolute values and to the sum of the squared
values:

Lambda is a shared penalization parameter while alpha sets the ratio
between L1 and L2 regularization in the Elastic Net Regularization.
Hence, we expect a hybrid~behavior between L1 and L2 regularization.

And that's happening: Though coefficients are cut, the cut is less
abrupt than the cut with lasso penalization alone.

~

A geometric perspective on regularization

The Lasso, Ridge and Elastic-net regression can also be viewed as a
constraint added to the optimization process. The Lasso~error
minimization can be rewritten:

And the ridge error minimization can be rewritten:

When written in this way, it's clear that Lasso restricts the
coefficients to a square shape (or an L1 sphere) which diagonals are
equal to 2t. The ridge error restricts the coefficients to a circle (or
an L2 sphere) of radius t.~

~

    \subsubsection{Grid Search Options.}\label{grid-search-options.}

    Grid search means you have a set of models (which differ from each other
in their parameter values, which lie on a grid). What you do is you then
train each of the models and evaluate it using cross-validation.

    A search consists of:

\begin{verbatim}
<li>an estimator (regressor or classifier such as&nbsp;<code>sklearn.svm.SVC()</code>);</li>
<li>a parameter space;</li>
<li>a method for searching or sampling candidates;</li>
<li>a cross-validation scheme; and</li>
<li>a&nbsp;<a href="http://scikit-learn.org/stable/modules/grid_search.html#gridsearch-scoring">score function</a>.</li>
\end{verbatim}


    % Add a bibliography block to the postdoc
    
    
    
    \end{document}
